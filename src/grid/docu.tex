%%This is a very basic article template.
%%There is just one section and two subsections.
\documentclass{article}

\title{Documentation 2DGridder}
\author{Uwe Hebbelmann, Sebastian Stock, Andre Schemschat}
\date{28.02.2008}

\begin{document}
\maketitle{}

\section{}
\subsection{About}
The program was written to convert three dimensional scans to two dimensional
maps. The scanparser processes the scan in five steps. First, the scans are read
using the scanparser based on the $scan_ io$ libraries. \newline
Afterwards the scans are transformated using the transformationmatrix out of
the frame-files. Within this step the y-correction and the min/max-Distance	
filter are applied.\newline
The next step convertes each scan to a scanGrid. The scanGrid ist divided into
cells, which are resolution cm in size. Each cell contains the probability of
it beeing free. During the process more filters (like taking the spotradius of 
the laser into consideration) are applied.\newline
If the grids are converted,
each grid can be written to disk using a gridWriter (by default its the ppmWriter. 
If you want to change that, just change the 2DGridder.cc) and then added to
the parcelmanager.\newline
The parcelmanager takes care of dividing the grid
into the corresponding parcels, creating new parcels if neccessary and saving/loading
them. If all scans are converted and added, the parcelmanager can write the
entire world (based on all created parcels) into a single 2dm file (see
formats), which can be viewed with the MapViewer. Furthermore it can export the
map as a huge grid (which takes a lot of memory!). This is especially needed
for the hough-method, but it also comes in handy for writing the map as a pgm
file.
The last step is to create polygon-maps out of the 2D-Map. The algorithm is
based on houghes.

\subsection{Resume-function}
The programm creates parcels during runtime. These parcels store the
information at a given position of possibly several scans. These parcels
are managed using parcelinfos and parcel-files. Each time the program quits,
it saves the created parcelinfos to a file called parcelinfo.conf. 

This file contains the name of the parcel and the offset. If the resume-flag
is set, the program tries to load the old parcelinfo file and reads the
information of already created parcels. New scans now access the same parcels
as old scans, which means, that they are added to the same worldmap.

This may be used, to iterativly call the 2DGridder with scans and add each
scan to the same worldmap, which is usefull for speeding up the process
( because the memory used by the scan is completly freed) or add new created
scans to an already processed batch of scans.

If the flag is set to true however and the parameters do not match (e.g.
parcelsize, resolution etc), the behavior will most certainly result in
an error.


This feature isn't yet fully tested and my contain bugs.


\section{Parameters}

\subsection{Usage}
The following text will give a short explanation the parameters used in the
2DGridder. Nearly all parameters are optional, the only parameter that is
required is scandir. \newline
Below there are a few samples how to invoke the programm using different
settings.
\newline
{\bf ./2DGridder ~/scans/terrainScan} \newline
This would process all Scans in the given folder with the standard arguments 
(so the scantype must be uos, otherwise it will fail). However this might not 
bring the best results.
\newline
\newline
{\bf ./2DGridder -fold -s1 -e3 -r1 -w dat/duj} \newline
This would process the scans one to three in the old scan format in the folder
./dat/duj with a resolution of one cell per cm without waypoints.

\subsection{scandir}
{\bf Default: \ None} \newline
{\bf Flag: \ None} \newline
The flag specifies the folder in which the programm searchs for the scans. The
structur of the folder needs to be according to the scantype (see flag
scantype), in order to successfully read the scandata and the frames
information.

\subsection{outputdir}
{\bf Default: \ './parcels'} \newline
{\bf Flag: \ -o Path} \newline
The outputdirectory specifies the folder where the worldmap, the viewpoints
and the polygon-file are saved to. The folder needs to exist, otherwise an
error is printed.

\subsection{start}
{\bf Default: \ 0} \newline
{\bf Flag: \ -s Nr} \newline
Start specifies the first scan to be read. By setting start and end, the range
of processed scan can be limited.
If startscan is smaller than the first scan available (e.g. -s0 but the first
scan starts with 1) an error will be printed

\subsection{end}
{\bf Default: \ -1} \newline
{\bf Flag: \ -e Nr} \newline
End specifies the last scan to be read. Similiar to start, it can reduce the
range of scans to process.
If the default value is used, all scans after start are processed, until no
more scans are found

\subsection{maxDistance}
{\bf Default: \ 2500} \newline
{\bf Flag: \ -m Nr} \newline
The range of the laser is often limited, so all scans above a specified
distance are insecure. To avoid this, all found points further than
maxDistance are not processed.
MaxDistance is given in cm, which means the default is 25 meters.

\subsection{minDistance}
{\bf Default: \ -1} \newline
{\bf Flag: \ -M nr} \newline
Similar to maxDistance, but ignoring Scans closer than the specified
distance. Per default the minimal distance is ignored.


\subsection{readInitial}
{\bf Default: false} \newline
{\bf Flag: \ -t} \newline
Read a file containing a initial transformation matrix.


\subsection{scantype}
{\bf Default: \ uos} \newline
{\bf Flag: \ -f {uos, uos_map, old, rts, rts_map, ifp, riegl, zahn, ply}} \newline
The scantype specifies the format of the scan to read. If the format does not
match the format of the scans to read, the process will fail (undetermined
abort)

\subsection{correctYAxis}
{\bf Default: \ false} \newline
{\bf Flag: \ -y} \newline
If set to true, the origin of all scans will be transformed to start at a
z-offset of 0. If you encounter problems with partial scans not getting into
the worldmap, try settings this flag and adjust the min- and maxRelevantHeight
a bit. (So no scans are left out, because they are ignored because of incorrect
scanvalues)

\subsection{maxRelevantHeigth}
{\bf Default: \ 50} \newline
{\bf Flag: \ -M Nr} \newline
Specifies the maximum height which is relevant for the robotor. All found
points with y-coordinates above this value are not processed.
The unit is cm, so the default is 0.5 meters.


\subsection{minRelevantHeight}
{\bf Default: \ 2} \newline
{\bf Flag: \ -m Nr} \newline
Specifies the minimum height which is relevant for the map. All points with an y-coordinate below this value are not processed. 
This parameter is useful for eliminating interferences from the floor in front of the roboter;

\subsection{resolution}
{\bf Default: \ 10} \newline
{\bf Flag: -r} \newline
The resolution specifies the scale of the grid. The minimum resolution is 1, which means, that each cm is represented by one cell. The default is 10, which means that 10 cm of scandata is aggregated to one cell.
The lower the resolution the higher the size of each grid and the longer it takes to convert the scan and the more memory is needed

\subsection{createWaypoints}
{\bf Default: \ true} \newline
{\bf Flag: \ -w} \newline
If this flag is set, then waypoints will NOT be created. If the flag ist not set,
then the process will create for each found and relevant
(see minRelevant / maxRelevantHeight or maxDistance) point
free points on the direct line from the viewpoint of the roboter and the point.


\subsection{createNeighbours}
{\bf Default: \ true} \newline
{\bf Flag: \ -n} \newline
If this flag is set, then neighbours will NOT be created. If the flag is not set,
than the process will check how far away the found point is and create neighbours 
based on the spotradius of the laser.


\subsection{parcelWidth}
{\bf Default: \ 100} \newline
{\bf Flag: \ -p Nr} \newline
The parcelwidth specifies the width of the parcels of the map. The finer the
scale is, the more often the parcels are written and read from the disk, but 
the worldmap will be smaller (since there are not as much useless information
in the world-file).

\subsection{parcelHeight}
{\bf Default: \ 100} \newline
{\bf Flag: -P Nr} \newline
See parcelWidth. 

\subsection{writeWorld}
{\bf Default: \ true} \newline
{\bf Flag: \ -d} \newline
If this flag is set, the world-file will NOT be created. By default the value is true, so all parcels are merged to one world-file. The world-file is written in the worldformat, which can be read by the MapViewer.

\subsection{writeGrids}
{\bf Default: \ false} \newline
{\bf Flag: \ -g} \newline
By default the value is false, so the single grid files are not written. If set
to true, each converted scan is written as a ppm-file to the disk.

\subsection{writeLines}
{\bf Default: \ false} \newline
{\bf Flag: \ -l} \newline
By default the value is false, so lines are not created. If the parameter is
set to true, the programm tries to extrapolate polygonlines out of the
worldmap using the hough-matrix.
This may take, dependend on the size of the map, very long.

\subsection{spotradius}
{\bf Default: \ 15} \newline
{\bf Flag: \ -a Nr} \newline
The spotradius is the deflection-factor of the laser on longer ranges. The
value given relates to the deflection at 30 meter (3000 cm), so the default
results in a factor of 1/200. The method to calculate the probability of
neighbours for each point is based on the deflection.

\subsection{count}
{\bf Default: \ 50} \newline
{\bf Flag: \ -c Nr} \newline
This flag specifies how many scans are loaded, process and added to the worldmap
until they are freed. Since the loaded scans and grids are not needed after
they are processed, they can be freed. This is achieved by processing count
scans in a row, and then freeing the not needed data.

\subsection{resume}
{\bf Default: false} \newline
{\bf Flag: \ -R} \newline
If this flag is set, the programm will load the parcels it has created
during the last run. See Resume-function for more information.

\section{formats}
This section is about the formats which can be written using the
2DGridder. For most formats it is only a short description, because most of
them are plain simple.

\subsection{PPM}
The ppm is a greyscale-picture format which is quite simple. It only needs the
size of the image and a value for each pixel. For more information, try google.

\subsection{gnuplot}
The gnuplot files are used to visualize the data in gnuplot (and maybe other
tools). The information is written as $x z value$ with value beeing -1 for no
information or $\in[0,1]$, each point a line.

\subsection{world (2dm)}
The worldformat is used to transport data between the 2Dgridder and the
MapViewer. The first line is the Version of the file. Mostly this is 1. The
second line specifies the resolution used in the grid. The third line specifies
the minimal and maximal found x and z values, in the order of $minX maxX minZ
maxZ$. The next line contains the position of the roboter during the first
scan (x and z coordinate). The rest of the file is similar to the gnuplot
format, which means $x z value$.

\subsection{Viewpoints}
The viewpoint-file contains a list of coordinates at which the roboter has
taken the scans. Each line contains a x and z value standing for the position.

\subsection{parcel}
The parcelformat is used to store the information of the parcels during
runtime. The parcels are stored as binary, so they can be stored and loaded
faster

\subsection{bor}
The bor-format is used to store the polygion-files created out of the map.

\end{document}
